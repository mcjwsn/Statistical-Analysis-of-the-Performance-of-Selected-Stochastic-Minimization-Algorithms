% Options for packages loaded elsewhere
\PassOptionsToPackage{unicode}{hyperref}
\PassOptionsToPackage{hyphens}{url}
%
\documentclass[
]{article}
\usepackage{amsmath,amssymb}
\usepackage{iftex}
\ifPDFTeX
  \usepackage[T1]{fontenc}
  \usepackage[utf8]{inputenc}
  \usepackage{textcomp} % provide euro and other symbols
\else % if luatex or xetex
  \usepackage{unicode-math} % this also loads fontspec
  \defaultfontfeatures{Scale=MatchLowercase}
  \defaultfontfeatures[\rmfamily]{Ligatures=TeX,Scale=1}
\fi
\usepackage{lmodern}
\ifPDFTeX\else
  % xetex/luatex font selection
\fi
% Use upquote if available, for straight quotes in verbatim environments
\IfFileExists{upquote.sty}{\usepackage{upquote}}{}
\IfFileExists{microtype.sty}{% use microtype if available
  \usepackage[]{microtype}
  \UseMicrotypeSet[protrusion]{basicmath} % disable protrusion for tt fonts
}{}
\makeatletter
\@ifundefined{KOMAClassName}{% if non-KOMA class
  \IfFileExists{parskip.sty}{%
    \usepackage{parskip}
  }{% else
    \setlength{\parindent}{0pt}
    \setlength{\parskip}{6pt plus 2pt minus 1pt}}
}{% if KOMA class
  \KOMAoptions{parskip=half}}
\makeatother
\usepackage{xcolor}
\usepackage[margin=1in]{geometry}
\usepackage{graphicx}
\makeatletter
\def\maxwidth{\ifdim\Gin@nat@width>\linewidth\linewidth\else\Gin@nat@width\fi}
\def\maxheight{\ifdim\Gin@nat@height>\textheight\textheight\else\Gin@nat@height\fi}
\makeatother
% Scale images if necessary, so that they will not overflow the page
% margins by default, and it is still possible to overwrite the defaults
% using explicit options in \includegraphics[width, height, ...]{}
\setkeys{Gin}{width=\maxwidth,height=\maxheight,keepaspectratio}
% Set default figure placement to htbp
\makeatletter
\def\fps@figure{htbp}
\makeatother
\setlength{\emergencystretch}{3em} % prevent overfull lines
\providecommand{\tightlist}{%
  \setlength{\itemsep}{0pt}\setlength{\parskip}{0pt}}
\setcounter{secnumdepth}{-\maxdimen} % remove section numbering
\usepackage{xcolor}
\ifLuaTeX
  \usepackage{selnolig}  % disable illegal ligatures
\fi
\usepackage{bookmark}
\IfFileExists{xurl.sty}{\usepackage{xurl}}{} % add URL line breaks if available
\urlstyle{same}
\hypersetup{
  pdftitle={Mój pierwszy dokument R Markdown},
  pdfauthor={Twoje Imię},
  hidelinks,
  pdfcreator={LaTeX via pandoc}}

\title{Mój pierwszy dokument R Markdown}
\author{Twoje Imię}
\date{2023-10-10}

\begin{document}
\maketitle

\section{Cel projektu}\label{cel-projektu}

Projekt polega na prostym opracowaniu statystycznym wyników porównania
działania wybranych algorytmów minimalizacji stochastycznej.
Zdecydowaliśmy się do porównania użyć następujących algorytmów:

\begin{itemize}
\tightlist
\item
  \textcolor{green}{Poszukiwanie przypadkowe (Pure Random Search, PRS)}
\item
  \textcolor{red}{Metoda wielokrotnego startu (multi-start, MS)}
\end{itemize}

\subsection{Opis algorytmów}\label{opis-algorytmuxf3w}

\subsubsection{Poszukiwanie przypadkowe (Pure Random Search,
PRS)}\label{poszukiwanie-przypadkowe-pure-random-search-prs}

Algorytm PRS polega na losowym przeszukiwaniu przestrzeni rozwiązań, w
której minimalizowana funkcja jest zdefiniowana. Działa w następujący
sposób:

\begin{enumerate}
\def\labelenumi{\arabic{enumi}.}
\item
  \textbf{Losowanie punktów}: Losujemy kolejne punkty w przestrzeni
  poszukiwań z rozkładu jednostajnego. Jeżeli dziedzina poszukiwań jest
  kostką wielowymiarową, to każdą współrzędną punktu losujemy z
  odpowiedniego jednowymiarowego rozkładu jednostajnego.\\
  Na przykład, jeśli dziedzina poszukiwań to kostka trójwymiarowa
  \([0,1] \times [-2,2] \times [100,1000]\), losowanie współrzędnych
  wygląda następująco:

  \begin{itemize}
  \tightlist
  \item
    pierwsza współrzędna: \(U(0,1)\),
  \item
    druga współrzędna: \(U(-2,2)\),
  \item
    trzecia współrzędna: \(U(100,1000)\).
  \end{itemize}
\item
  \textbf{Porównanie wartości funkcji}: Wartość funkcji w każdym
  wylosowanym punkcie porównujemy z aktualnie zapamiętanym minimum.
  Jeśli wartość funkcji w nowym punkcie jest mniejsza, zapamiętujemy ten
  punkt jako nowe minimum.
\item
  \textbf{Wynik}: Wartość funkcji w ostatnim zapamiętanym punkcie
  stanowi wynik algorytmu.
\end{enumerate}

\begin{center}\rule{0.5\linewidth}{0.5pt}\end{center}

\subsubsection{Metoda wielokrotnego startu (Multi-Start,
MS)}\label{metoda-wielokrotnego-startu-multi-start-ms}

Algorytm MS łączy losowe przeszukiwanie przestrzeni z metodami
optymalizacji lokalnej. Jego kroki są następujące:

\begin{enumerate}
\def\labelenumi{\arabic{enumi}.}
\item
  \textbf{Losowanie punktów}: Podobnie jak w PRS, losujemy zadany zbiór
  punktów startowych z rozkładu jednostajnego w przestrzeni poszukiwań.
\item
  \textbf{Uruchomienie optymalizacji lokalnej}: Dla każdego wylosowanego
  punktu startowego uruchamiana jest metoda optymalizacji lokalnej (np.
  metoda L-BFGS-B zaimplementowana w funkcji \texttt{optim()} w R).
\item
  \textbf{Porównanie wyników}: Dla każdego startu zapisujemy wartość
  funkcji w zwróconym punkcie lokalnego minimum. Wynikiem algorytmu jest
  minimalna wartość funkcji spośród wszystkich punktów końcowych.
\end{enumerate}

\begin{center}\rule{0.5\linewidth}{0.5pt}\end{center}

Oba algorytmy można stosować w zadanej dziedzinie, różnią się jednak
podejściem. PRS polega na losowym próbkowaniu całej przestrzeni,
natomiast MS wykorzystuje optymalizację lokalną, aby dokładniej zbadać
okolice losowo wybranych punktów startowych.

Do porównania należy wybrać dwie z funkcji dostępnych w pakiecie
\texttt{smoof}, które są skalarne (single-objective) i mają wersje dla
różnej liczby wymiarów (akceptują parametr \texttt{dimensions}).

\end{document}
